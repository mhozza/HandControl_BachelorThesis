%\cleardoublepage
\phantomsection
\addcontentsline{toc}{chapter}{Úvod}
\chapter*{Úvod}\label{chap:intro}
Notebooky sú často využívané na prezentácie, či už prednášok spracovaných v~rôznych aplikáciách, alebo jednoduchého zobrazenia fotiek fotiek. Pri prezentáciách sa často využíva diaľkový ovládač, aby prezentujúci nemusel sedieť pri počítači alebo k~nemu stále chodiť.

V~dnešnej dobe väčšina notebookov obsahuje webovú kameru, takže sa ponúka otázka, či sa nedá kamera využiť na elimináciu potreby ovládača. Počítač by sa mohol ovládať pomocou pohybu ruky, ktorý by sa snímal webkamerou.

V~čase, keď vznikol nápad písať túto prácu\footnote{v roku 2010} sa takéto riešenia v~praxi nevyužívali, hoci niečim podobným sa už zaoberali viacerí. Nenašli sme však žiadne dokončené riešenia, ktoré by boli voľne prístupné.

Ešte pred dokončením tejto práce bol uvedený na trh inteligentný televízor, ktorý bolo možné ovládať kombináciou hlasu a pohybu ruky, čo naznačuje, že táto myšlienka nie je zlá. 

Narozdiel od spracovania v~televízore, kde sa ruka používala na ovládanie kurzora myši, my budeme pohyby ruky interpretovať ako gestá a následne prekladať na príkazy pre počítač - pomocou simulácie stlačení klávesov.

Práca je zameraná na problém rozpoznania pohybujúcej sa ruky v~obraze, ktorý je získaný z~webkamery. Rozpoznávanie realizujeme pomocou neurónových sietí. Popíšeme  návrh architektúry siete a rôzne spôsoby predspracovania dát, aby sme dosiahli čo najlepšiu úspešnosť rozpoznávania. Ukážeme, aké prístupy môžu pomôcť a čo sa stane ak do siete pridáme rekurentné spojenia. 

Nakoniec popíšeme implementáciu aplikácie, ktorá bude rozpoznávať jednoduché gestá ruky a bude umožňovať ovládať iné aplikácie rozpoznanými gestami.

Výsledný produkt sa bude dať použiť napríklad na ovládanie prezentačnej aplikácie alebo prehliadača fotiek, či prehrávača videa.