\chapter{Technológie}\label{chap:tech}

V tejto kapitole sa budem venovať použitým technológiám.
\bigskip

Vhodnou platformou pre vývoj aplikácií, ktoré pracujú s perifériami je linux, pretože sú dostupné otvorené ovládače a vynikajúca podpora.
Keďže linux je písaný hlavne v jazyku C, prevažná väčšina knižníc pre linux sa dodáva aj s hlavičkovými súbormi pre jazyk C a C++.
Preto je jazyk C++ veľmi vhodný pre programovanie pre túto platformu. Navyše je objektovo orientovaný a umožnuje jednoduché použitie komplexných dátových štruktúr.

Programovanie nám uľahčia už hotové knižnice. Využijeme knižnicu Qt(multiplatformová knižnica umožnujúca vytváranie okien, správu vlákien a mnohé iné),
video for linux 2(v4l2 - na prácu z webkamerou), fftw3 (rýchla knižnica počítajúca fourierovu transformáciu) a Xtst(knižnica ktorá okrem iného umožnuje simulovanie kláves v X servri).
