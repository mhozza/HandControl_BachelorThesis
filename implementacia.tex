\chapter{Implementácia}\label{chap:implementation}

V tejto kapitole predstavíme triedy a popíšeme implementačné detaily 
\bigskip

\section{Základné triedy}
\subsection{HCImage, ColorImage, GrayScaleImage}

Pre účely aplikácie potrebujeme špecifickú triedu na obrázky, ktorá spĺňa nasledujúce vlastnosti:
\begin{itemize}
\item Rýchly prístup k jednotlivým pixlom.
\item Vystrihnutie(crop)
\item Škálovanie
\item Maskovanie
\end{itemize}
Okrem toho potrebujeme aby obsahovala aj pomocné metódy, ktoré sú potrebné na niektoré algoritmy a testovanie.

Trieda HCImage, je abstraktná trieda ktorá implementuje základné veci, ktoré sú nezávislé na type pixelu. Od tejto triedy potom dedia triedy GrayScaleImage a ColorImage. Tieto implementujú metódy, ktoré sú závislé na type pixla. 

%V ďalšom texte sa budeme venovať jednotliv

\subsubsection{Vystrihnutie}
Na vystrihnutie slúži metóda {\tt copy(QRect r)}, ktorá berie ako parameter obdĺžnik, ktorého obsah potom vráti ako výsledok. 

\subsubsection{Škálovanie}
Metóda {\tt scale(unsigned w, unsigned h)} využíva bilineárnu interpoláciu. Vypočíta súradnice nového bodu a pomocou susedných bodov z pôvodného obrázka vypočíta farbu.

Bilineárna interpolácia má niekoľko výhod. Je stále dosť rýchla, pričom eliminuje kostrbatosť, ktorá môže mať výrazný vplyv pri použití Fourierovej transformácie.

\subsubsection{Maskovanie}
Maskovanie použijeme na odstránenie nepodstatných častí obrázka, aby neovplyvňovali neurónovú sieť. Metóda {\tt mask(HCImage *mask, bool invert = false)} využíva bitové operátory na urýchlenie výpočtu. Každý pixel skombinuje s pixlom masky. 

\section{Získanie obrazu z webkamery}
Na získanie obrazu z webkamery využijeme knižnicu 