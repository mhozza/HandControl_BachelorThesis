\chapter{Implementácia}\label{chap:implementation}

V~tejto kapitole predstavíme triedy a popíšeme implementačné detaily 
\bigskip

\section{Základné triedy}
\subsection{HCImage, ColorImage, GrayScaleImage}

Pre účely aplikácie potrebujeme špecifickú triedu na obrázky, ktorá spĺňa nasledujúce vlastnosti:
\begin{itemize}
\item Rýchly prístup k~jednotlivým pixlom.
\item Vystrihnutie(crop)
\item Škálovanie
\item Maskovanie
\item Floodfill
\end{itemize}
Okrem toho potrebujeme aby obsahovala aj pomocné metódy, ktoré sú potrebné na niektoré algoritmy a testovanie konverzia do rôznych formátov, ukladanie na disk a~iné.

Trieda HCImage, je abstraktná trieda ktorá implementuje základné veci, ktoré sú nezávislé na type pixelu. Od tejto triedy potom dedia triedy GrayScaleImage a ColorImage. Tieto implementujú metódy, ktoré sú závislé na type pixla. 

%V ďalšom texte sa budeme venovať jednotliv

\subsubsection{Vystrihnutie}
Na vystrihnutie slúži metóda {\tt copy(QRect r)}, ktorá berie ako parameter obdĺžnik, ktorého obsah potom vráti ako výsledok. 

\subsubsection{Škálovanie}
Metóda {\tt scale(unsigned w, unsigned h)} využíva bilineárnu interpoláciu. Vypočíta súradnice nového bodu a pomocou susedných bodov z~pôvodného obrázka vypočíta farbu.

Bilineárna interpolácia má niekoľko výhod. Je stále dosť rýchla, pričom eliminuje kostrbatosť, ktorá môže mať výrazný vplyv pri použití Fourierovej transformácie.

\subsubsection{Maskovanie}
Maskovanie použijeme na odstránenie nepodstatných častí obrázka, aby neovplyvňovali neurónovú sieť. Metóda {\tt mask(HCImage *mask, bool invert = false)} využíva bitové operátory na urýchlenie výpočtu. Každý pixel skombinuje s~pixlom masky. 

\subsubsection{Floodfill}
Floodfill si môžeme predstaviť ako "čarovnú paličku"\footnote{známu z~grafických programov ako je GIMP, či Photoshop}, alebo nástroj na selekciu susedných pixlov s~podobnými farbami. Floodfill používame na získanie masky v~metóde {\tt getAdaptiveFloodFillSelectionMask(int sx, int sy, int treshold, float originalFactor, float changeFactor)}. Metóda berie nekoľko parametrov - pozíciu bodu z~ktorého má floodfill začať, hranicu podobnosti farieb, vplyv originálnej farby a zmeny. Referenčná farba sa získa ako priemer začiatočného bodu a jeho okolitých bodov. Algortimus funguje ako prehľadávanie do šírky, pričom do susedného pixla pôjde len v~prípade, že ich rozdiel je menší ako hranica. V~prípade farebného obrázka sa tento rozdiel počíta po zložkách a menší musí byť každý z~nich. Pri prejdení do nasledujúceho pixla sa upraví referenčná farba podľa nasledujúceho vzorca: $refcolor = reference \cdot originalFactor + (refcolor\cdot changeFactor+color \cdot (1-changeFactor)) \cdot (1-originalFactor)$, kde $reference$ je začiatočná referenčná farba, $refcolor$ je aktuálna referenčná farba a $color$ je farba pixla.

\section{Získanie obrazu z~webkamery}
Na získanie obrazu z~webkamery využijeme knižnicu 