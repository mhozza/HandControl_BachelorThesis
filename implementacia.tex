\chapter{Implementácia}\label{chap:implementation}

V~tejto kapitole predstavíme triedy a popíšeme implementačné detaily jednotlivých častí algoritmu.

Budeme sa venovať základným triedam, riešeniam elementárnych vecí, detailom spacovania obrazu, rozpoznávaniu ruky a gesta. Podrobne si popíšeme implementáciu neurónových sietí.
\bigskip

\section{Triedy pre obrázky}

Pre účely aplikácie potrebujeme špecifickú triedu na obrázky, ktorá spĺňa nasledujúce vlastnosti:
\begin{itemize}
\item Rýchly prístup k~jednotlivým pixlom.
\item Vystrihnutie(crop)
\item Škálovanie
\item Maskovanie
\item Floodfill
\end{itemize}
Okrem toho potrebujeme aby obsahovala aj pomocné metódy, ktoré sú potrebné na niektoré algoritmy a testovanie konverzia do rôznych formátov, ukladanie na disk a~iné.

Trieda \textit{HCImage}, je abstraktná trieda ktorá implementuje základné veci, ktoré sú nezávislé na type pixelu. Od tejto triedy potom dedia triedy \textit{GrayScaleImage} a \textit{ColorImage}. Tieto implementujú metódy, ktoré sú závislé na type pixla. 

%V ďalšom texte sa budeme venovať jednotliv
\subsubsection{Vystrihnutie}
Na vystrihnutie slúži metóda {\tt copy(QRect r)}, ktorá berie ako parameter obdĺžnik, ktorého obsah potom vráti ako výsledok. 

\subsubsection{Škálovanie}
Metóda {\tt scale(unsigned w, unsigned h)} využíva bilineárnu interpoláciu. Vypočíta súradnice nového bodu a pomocou susedných bodov z~pôvodného obrázka vypočíta farbu.

Bilineárna interpolácia má niekoľko výhod. Je stále dosť rýchla, pričom eliminuje kostrbatosť, ktorá môže mať výrazný vplyv pri použití Fourierovej transformácie.

\subsubsection{Maskovanie}
Maskovanie použijeme na odstránenie nepodstatných častí obrázka, aby neovplyvňovali neurónovú sieť. Metóda {\tt mask(HCImage *mask, bool invert = false)} využíva bitové operátory na urýchlenie výpočtu. Každý pixel skombinuje s~pixlom masky. 

\subsubsection{Floodfill}
Floodfill si môžeme predstaviť ako "čarovnú paličku"\footnote{známu z~grafických programov ako je GIMP, či Photoshop}, alebo nástroj na selekciu susedných pixlov s~podobnými farbami. Floodfill používame na získanie masky v~metóde {\tt getAdaptiveFloodFillSelectionMask(int sx, int sy, int treshold, float originalFactor, float changeFactor)}. Metóda berie nekoľko parametrov - pozíciu bodu z~ktorého má floodfill začať, hranicu podobnosti farieb, vplyv originálnej farby a zmeny. Referenčná farba sa získa ako priemer začiatočného bodu a jeho okolitých bodov. Algortimus funguje ako prehľadávanie do šírky, pričom do susedného pixla pôjde len v~prípade, že ich rozdiel je menší ako hranica. V~prípade farebného obrázka sa tento rozdiel počíta po zložkách a menší musí byť každý z~nich. Pri prejdení do nasledujúceho pixla sa upraví referenčná farba podľa nasledujúceho vzorca: $refcolor = reference \cdot originalFactor + (refcolor\cdot changeFactor+color \cdot (1-changeFactor)) \cdot (1-originalFactor)$, kde $reference$ je začiatočná referenčná farba, $refcolor$ je aktuálna referenčná farba a $color$ je farba pixla.

\section{Získanie obrazu z~webkamery}
Na získanie obrazu z~webkamery používam triedu, ktorá pochádza z programu \textit{Kapture} \cite{kapture} a ktorú sme upravili pre potreby využitia v našej aplikácii. Webkamera vracia obraz vo formáte MJPEG(Motion JPEG), teda postupnosť obrázkov vo formáte JPEG. Úlohou našej triedy je vždy keď je buffer pripravený, zobrať pretransformovať obrázok do niektorej z tried na obrázky. JPEG kóduje farby vo formáte YUV, tieto teda musíme pretransformovať do RGB. V prípade čiernobieleho obrázka iba zoberieme zložku Y.

\section{Spracovanie obrazu}
\subsection{Trieda ImageProcessor}
Trieda \textit{ImageProcessor} má na starosti predspracovanie obrazu z webkamery, jeho segmentáciu a posunutie na ďalšie spracovanie triede \textit{HandRecognizer}. \textit{ImageProcessor} využíva vlákna na paralelizáciu úloh a jeho práca je rozdelená do niekoľkých krokov, ktoré nemôžu byť vykonané paralelne. Hlavná metóda je metóda {\tt processImage (const GrayScaleImage \&image, const ColorImage \&colorimg)}, ktorá spravuje vlákna a spúšťa jednotlivé kroky. 

Spúštanie vláken je realizované pomocou funkcie {\tt QtConcurrent::run(Function function, ...)} z knižnice \textit{Qt}, ktorá umožňuje spustiť funkciu s danými parametrami v novom vlákne. Jednotlivé metódy sú upravené tak, aby vedeli upravovať aj len určitú časť obrázka. Spúšťajú sa v rôznych vláknach s rôznymi časťami obrázka. %Kritické časti sme ohraničili mutexami.

\subsubsection{Krok 1: Príprava}
Príprava spočíva vo vytvorení rozdielového obrázka. Pamätáme si pôvodný obrázok a dostaneme nový. Spočítame rozdiel a zapamätaný obrázok potom nahradíme novým.

Na začiatku je pôvodný obrázok celý čierny a v rozdieli toho bude veľa. To nám však nevadí. 

\subsubsection{Krok 2: Rozpitie}
Označíme si dĺžku strany štvorca $k$.

Triviálny algoritmus rozpitia - pre každý pixel vyrobíme okolo neho čierny štvorček - je časovo náročný - má zložitosť $O(k^2.w.h)$. Preto potrebujeme efektívnejší. 
Efektívny algoritmus rozpitia treba robiť v 2 krokoch. Najskôr sa rozpije v $X$-ovom smere a potom v $Y$-ovom.

Algoritmus pre $X$-ový smer funguje nasledovne: Ideme postupne po riadkoch a pre každý riadok si pamätáme pokiaľ máme kresliť čiernu.
Na začiatku je to 0. Pozeráme sa vždy o $\frac{k}{2}$ pixlov ďalej ako kreslíme.
Vždy keď vidíme pixel inej farby ako bielej, tak hranicu pokiaľ máme kresliť čiernu posunieme na hodnotu o $k$ väčšiu ako je $y$ súradnica pixla ktorý kreslíme. Pokiaľ sme pred hranicou, kreslíme čiernu, inak bielu. Tento algoritmus má zložitosť $O(w.h)$ v praxi pre $k=11$ asi $5\times$ rýchlejší.  

Keďže riadky sa spracovávajú nezávisle, tak sa dá obrázok rozdeliť na niekoľko častí, ktoré sa dajú spracovať paralelne.

Pre $Y$-ový smer je to analogické. 

\subsubsection{Krok 3: Segmentácia}
Ako sme už spomenuli v kapitole \ref{subsubsect:segment}, v časti segmentácia, obrázok si reprezentujeme ako graf, na ktorý následne použijeme algoritmus prehľadávania do šírky. Algoritmus postupne spustíme z každého bodu obrázka, ktorý nie je prázdny a zároveň sme ho predtým ešte žiadnym predošlým behom algoritmu nenavštívili. Každý beh nám vráti obdĺžnik, ktorý ak spĺňa veľkostné obmedzenia, tak je vložený do fronty na ďalšie spracovanie. 

Táto časť je realizovaná len jedným vláknom, keďže obrázok nie je možné vhodne rozdeliť.

\section{Rozpoznanie ruky}

Rozpoznanie ruky má nastarosti trieda \textit{HandRecognizer}. Beží naraz v niekoľkých vláknach - každé vlákno spracováva len jeden segment.
Trieda \textit{HandRecognizer} vyberie z fronty obdĺžnik čakajúci na spracovanie. Podľa rozmerov a pozície vystrihne príslušnú časť obrázka.


\section{Rozpoznanie gesta}


\section{Simulácia stlačenia kláves}
Knižnica \textit{Xtst} poskytuje funkciu {\tt XTestFakeKeyEvent}, ktorá nám umožnuje simulovať stlačenie klávesy. Do nej vložíme pointer na display, kód klávesy, či je stlačená alebo nie(funkciu treba použiť 2x - raz na stlačenie a raz na pustenie klávesy) a aktuálny čas.