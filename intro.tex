%\cleardoublepage
\phantomsection
\addcontentsline{toc}{chapter}{Úvod}
\chapter*{Úvod}\label{chap:intro}
{\color{red}
Notebooky sú často využívané na prezentácie, či už na nejakej prednáške, alebo prezentácie fotiek. Pri prezentovaní sa často využíva diaľkový ovládač, aby prezentujúci nemusel sedieť pri počítači alebo k nemu stále chodiť.

V dnešnej dobe väčšina notebookov obsahuje web-kameru, takže vzniká otázka, či sa nedá kamera využiť na elimináciu potreby ovládača. Počítač by sa mohol ovládať pomocou pohybu ruky, ktorý by sa snímal web-kamerou.

V tejto práci sa budem zaoberať 2 prístupmi k rozpoznávaniu a budem porovnávať ich úspešnosť. Prvým z nich je oddelený prístup, kde budem zvlášť rozpoznávať ruku pomocou doprednej neurónovej siete a zvlášť gestá nakreslené touto rukou.
V druhom budem rozpoznávať gesto z postupnosti obrazov pomocou rekurentnej neurónovej siete. Okrem toho sa budem venovať predspracovaniu obrazu rôznymi metódami tak, aby som zlepšil kvalitu rozpoznávania.

Výsledný produkt sa bude dať použiť na transformáciu gest na klávesové skratky a tým na ovládanie prezentačnej aplikácie.
}

% \section{Motivácia}
% 
% Motiváciou mojej práce bolo spraviť produkt, ktorý umožní vzdialené ovládanie počítača, bez pomoci ovládača, alebo iného externého zariadenia. Produkt by bol použiteľný napríklad pri prezentáciach.
% 
% \section{Ciele}
% 
% Cieľom práce je pomocou počítačových neurónových sietí rozpoznávať jednoduché dynamické gestá prezentované rukou a zaznamenané webovou kamerou. Porovnať dva prístupy k rozpoznávaniu: cez doprednú a rekurentnú neurónovú sieť. 
% Realizovaný rozpoznávač prepojiť na ovládanie aplikácií.
% 
