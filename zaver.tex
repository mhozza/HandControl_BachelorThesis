%\cleardoublepage
\chapter*{Záver}\label{chap:conc}
\phantomsection
\addcontentsline{toc}{chapter}{Záver}

V práci sme popísali návrh aplikácie pre ovládanie počítača pomocou gest ruky. Popísali sme si algoritmus rozpoznávania gesta a podrobnejšie sme sa zaoberali jeho jednotlivými časťami. Stručne sme si vysvetlili, čo sú umelé neurónové siete a ako ich použijeme na rozpoznanie ruky v obraze. Zaoberali sme sa aj technickými časťami aplikácie, ako je preklad gesta na stlačenia kláves a získavanie obrazu z webovej kamery.

Navrhli sme vhodnú architektúru umelej neurónovej siete na rozpoznanie ruky a vyskúšali sme rôzne metódy predspracovania. Porovnali sme úspešnosti jednotlivých prístupov a vybrali sme ti, ktoré sa ukázali byť nápomocné. Nakoniec sme si popísali implementáciu aplikácie a rozobrali sme niektoré problémy, ktoré sa vyskytli pri jej implementácii.

Podarilo sa nám ukázať, že Fourierova transformácia pomáha na zvýšenie úspešnosti, a že rozdielový obraz, ktorý vznikne rozdielom dvoch po sebe idúcich obrázkov a obsahuje obrysy pohybujúcich sa objektov je vhodným zdrojom informácie pre neurónovú sieť. 

Ukázali sme, že umelé neurónové siete sa dajú použiť na úlohu rozpoznávania ruky. Natrénovali sme neurónovú sieť, ktorá má úspešnosť 67,88 \% na testovacej vzorke a otestovali sme aplikáciu v praxi. Aplikácia je v praxi použiteľná, rozpoznávanie gest dosahuje úspešnosť približne 80\%.

Momentálne aplikácia rozpoznáva 4 základné gestá -- pohyb rukou doprava, doľava, hore a dole. Snažili sme sa ju zoptimalizovať a na bežných systémoch funguje plynule a bez problémov. Oproti komerčným riešeniam, ktoré sa dodávajú aj so špecifickým hardvérom má aplikácia nevýhodu v nízkej kvalite obrazu z kamery a to, že kamera je len jedna a obraz je dvojrozmerný -- komerčné riešenia majú často 2 kamery a senzor hĺbky, vďaka čomu majú verný 3D obraz scény. Aj napriek tomu však naša aplikácia funguje v bežných svetelných podmienkach celkom dobre.

Aplikácia poskytuje ešte mnoho priestoru na vylepšenie. Dal by sa použiť napríklad \textit{Kalmanov filter} na zvýšenie presnosti gesta a elimináciu zlých rozhodnutí siete, zlepšiť rozpoznávanie gest a pridať ďalšie, aj komplexnejšie gestá. Rozpoznávanie sa dá napojiť aj na kvalitnejší algoritmus sledovania objektov. V budúcnosti by sme chceli vyskúšať aj použitie \textit{Skrytých Markvovych modelov} namiesto neurónových sietí.

