\chapter{Platforma, jazyk a knižnice}

V~tejto kapitole sa budeme venovať použitým technológiám, použitému jazyku a dôvodom, prečo sme si ich vybrali.
\bigskip

Vhodnou platformou pre vývoj aplikácií, ktoré pracujú s~perifériami je linux, pretože sú dostupné otvorené ovládače a vynikajúca podpora. S~perifériami sa pracuje v~linuxe velmi pohodlne, každé zariadenie má vlastný súbor ktorý sa nachádza v~{\tt /dev/}.
Keďže linux je písaný hlavne v~jazyku C, prevažná väčšina knižníc pre linux sa dodáva aj s~hlavičkovými súbormi pre jazyk C a C++.
Preto je jazyk C++ veľmi vhodný pre programovanie pre túto platformu. Navyše je objektovo orientovaný a umožnuje jednoduché použitie komplexných dátových štruktúr.

Programovanie nám uľahčia už hotové knižnice. Využijeme knižnicu \textit{Qt} (multiplatformová knižnica umožnujúca vytváranie okien, správu vlákien a mnohé iné),
video for linux 2(\textit{v4l2} - na prácu z~webkamerou), \textit{fftw3} (rýchla knižnica počítajúca fourierovu transformáciu) a \textit{Xtst}(knižnica ktorá okrem iného umožnuje simulovanie kláves v~X servri).

Aplikácia využíva vlákna, na paralelizáciu a urýchlenie niektorých výpočtov.
